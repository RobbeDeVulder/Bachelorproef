%===============================================================================
% LaTeX sjabloon voor de bachelorproef toegepaste informatica aan HOGENT
% Meer info op https://github.com/HoGentTIN/latex-hogent-report
%===============================================================================

\documentclass[dutch,dit,thesis]{hogentreport}

% TODO:
% - If necessary, replace the option `dit`' with your own department!
%   Valid entries are dbo, dbt, dgz, dit, dlo, dog, dsa, soa
% - If you write your thesis in English (remark: only possible after getting
%   explicit approval!), remove the option "dutch," or replace with "english".

\usepackage{lipsum} % For blind text, can be removed after adding actual content

%% Pictures to include in the text can be put in the graphics/ folder
\graphicspath{{graphics/}}

%% For source code highlighting, requires pygments to be installed
%% Compile with the -shell-escape flag!
\usepackage[section]{minted}
%% If you compile with the make_thesis.{bat,sh} script, use the following
%% import instead:
%% \usepackage[section,outputdir=../output]{minted}
\usemintedstyle{solarized-light}
\definecolor{bg}{RGB}{253,246,227} %% Set the background color of the codeframe

%% Change this line to edit the line numbering style:
\renewcommand{\theFancyVerbLine}{\ttfamily\scriptsize\arabic{FancyVerbLine}}

%% Macro definition to load external java source files with \javacode{filename}:
\newmintedfile[javacode]{java}{
    bgcolor=bg,
    fontfamily=tt,
    linenos=true,
    numberblanklines=true,
    numbersep=5pt,
    gobble=0,
    framesep=2mm,
    funcnamehighlighting=true,
    tabsize=4,
    obeytabs=false,
    breaklines=true,
    mathescape=false
    samepage=false,
    showspaces=false,
    showtabs =false,
    texcl=false,
}

% Other packages not already included can be imported here

%%---------- Document metadata -------------------------------------------------
% TODO: Replace this with your own information
\author{Ernst Aarden}
\supervisor{Dhr. F. Van Houte}
\cosupervisor{Mevr. S. Beeckman}
\title[Optionele ondertitel]%
    {Titel van de bachelorproef}
\academicyear{\advance\year by -1 \the\year--\advance\year by 1 \the\year}
\examperiod{1}
\degreesought{\IfLanguageName{dutch}{Professionele bachelor in de toegepaste informatica}{Bachelor of applied computer science}}
\partialthesis{false} %% To display 'in partial fulfilment'
%\institution{Internshipcompany BVBA.}

%% Add global exceptions to the hyphenation here
\hyphenation{back-slash}

%% The bibliography (style and settings are  found in hogentthesis.cls)
\addbibresource{bachproef.bib}            %% Bibliography file
\addbibresource{../voorstel/voorstel.bib} %% Bibliography research proposal
\defbibheading{bibempty}{}

%% Prevent empty pages for right-handed chapter starts in twoside mode
\renewcommand{\cleardoublepage}{\clearpage}

\renewcommand{\arraystretch}{1.2}

%% Content starts here.
\begin{document}

%---------- Front matter -------------------------------------------------------

\frontmatter

\hypersetup{pageanchor=false} %% Disable page numbering references
%% Render a Dutch outer title page if the main language is English
\IfLanguageName{english}{%
    %% If necessary, information can be changed here
    \degreesought{Professionele Bachelor toegepaste informatica}%
    \begin{otherlanguage}{dutch}%
       \maketitle%
    \end{otherlanguage}%
}{}

%% Generates title page content
\maketitle
\hypersetup{pageanchor=true}

%%=============================================================================
%% Voorwoord
%%=============================================================================

\chapter*{\IfLanguageName{dutch}{Woord vooraf}{Preface}}%
\label{ch:voorwoord}

%% TODO:
%% Het voorwoord is het enige deel van de bachelorproef waar je vanuit je
%% eigen standpunt (``ik-vorm'') mag schrijven. Je kan hier bv. motiveren
%% waarom jij het onderwerp wil bespreken.
%% Vergeet ook niet te bedanken wie je geholpen/gesteund/... heeft

\lipsum[1-2]
%%=============================================================================
%% Samenvatting
%%=============================================================================

% TODO: De "abstract" of samenvatting is een kernachtige (~ 1 blz. voor een
% thesis) synthese van het document.
%
% Een goede abstract biedt een kernachtig antwoord op volgende vragen:
%
% 1. Waarover gaat de bachelorproef?
% 2. Waarom heb je er over geschreven?
% 3. Hoe heb je het onderzoek uitgevoerd?
% 4. Wat waren de resultaten? Wat blijkt uit je onderzoek?
% 5. Wat betekenen je resultaten? Wat is de relevantie voor het werkveld?
%
% Daarom bestaat een abstract uit volgende componenten:
%
% - inleiding + kaderen thema
% - probleemstelling
% - (centrale) onderzoeksvraag
% - onderzoeksdoelstelling
% - methodologie
% - resultaten (beperk tot de belangrijkste, relevant voor de onderzoeksvraag)
% - conclusies, aanbevelingen, beperkingen
%
% LET OP! Een samenvatting is GEEN voorwoord!

%%---------- Nederlandse samenvatting -----------------------------------------
%
% TODO: Als je je bachelorproef in het Engels schrijft, moet je eerst een
% Nederlandse samenvatting invoegen. Haal daarvoor onderstaande code uit
% commentaar.
% Wie zijn bachelorproef in het Nederlands schrijft, kan dit negeren, de inhoud
% wordt niet in het document ingevoegd.

\IfLanguageName{english}{%
\selectlanguage{dutch}
\chapter*{Samenvatting}
\lipsum[1-4]
\selectlanguage{english}
}{}

%%---------- Samenvatting -----------------------------------------------------
% De samenvatting in de hoofdtaal van het document

\chapter*{\IfLanguageName{dutch}{Samenvatting}{Abstract}}

\lipsum[1-4]


%---------- Inhoud, lijst figuren, ... -----------------------------------------

\tableofcontents

% In a list of figures, the complete caption will be included. To prevent this,
% ALWAYS add a short description in the caption!
%
%  \caption[short description]{elaborate description}
%
% If you do, only the short description will be used in the list of figures

\listoffigures

% If you included tables and/or source code listings, uncomment the appropriate
% lines.
%\listoftables
%\listoflistings

% Als je een lijst van afkortingen of termen wil toevoegen, dan hoort die
% hier thuis. Gebruik bijvoorbeeld de ``glossaries'' package.
% https://www.overleaf.com/learn/latex/Glossaries

%---------- Kern ---------------------------------------------------------------

\mainmatter{}

% De eerste hoofdstukken van een bachelorproef zijn meestal een inleiding op
% het onderwerp, literatuurstudie en verantwoording methodologie.
% Aarzel niet om een meer beschrijvende titel aan deze hoofdstukken te geven of
% om bijvoorbeeld de inleiding en/of stand van zaken over meerdere hoofdstukken
% te verspreiden!

%%=============================================================================
%% Inleiding
%%=============================================================================

\chapter{\IfLanguageName{dutch}{Inleiding}{Introduction}}%
\label{ch:inleiding}

De inleiding moet de lezer net genoeg informatie verschaffen om het onderwerp te begrijpen en in te zien waarom de onderzoeksvraag de moeite waard is om te onderzoeken. In de inleiding ga je literatuurverwijzingen beperken, zodat de tekst vlot leesbaar blijft. Je kan de inleiding verder onderverdelen in secties als dit de tekst verduidelijkt. Zaken die aan bod kunnen komen in de inleiding~\autocite{Pollefliet2011}:

\begin{itemize}
  \item context, achtergrond
  \item afbakenen van het onderwerp
  \item verantwoording van het onderwerp, methodologie
  \item probleemstelling
  \item onderzoeksdoelstelling
  \item onderzoeksvraag
  \item \ldots
\end{itemize}

\section{\IfLanguageName{dutch}{Probleemstelling}{Problem Statement}}%
\label{sec:probleemstelling}

Uit je probleemstelling moet duidelijk zijn dat je onderzoek een meerwaarde heeft voor een concrete doelgroep. De doelgroep moet goed gedefinieerd en afgelijnd zijn. Doelgroepen als ``bedrijven,'' ``KMO's'', systeembeheerders, enz.~zijn nog te vaag. Als je een lijstje kan maken van de personen/organisaties die een meerwaarde zullen vinden in deze bachelorproef (dit is eigenlijk je steekproefkader), dan is dat een indicatie dat de doelgroep goed gedefinieerd is. Dit kan een enkel bedrijf zijn of zelfs één persoon (je co-promotor/opdrachtgever).

\section{\IfLanguageName{dutch}{Onderzoeksvraag}{Research question}}%
\label{sec:onderzoeksvraag}

Wees zo concreet mogelijk bij het formuleren van je onderzoeksvraag. Een onderzoeksvraag is trouwens iets waar nog niemand op dit moment een antwoord heeft (voor zover je kan nagaan). Het opzoeken van bestaande informatie (bv. ``welke tools bestaan er voor deze toepassing?'') is dus geen onderzoeksvraag. Je kan de onderzoeksvraag verder specifiëren in deelvragen. Bv.~als je onderzoek gaat over performantiemetingen, dan 

\section{\IfLanguageName{dutch}{Onderzoeksdoelstelling}{Research objective}}%
\label{sec:onderzoeksdoelstelling}

Wat is het beoogde resultaat van je bachelorproef? Wat zijn de criteria voor succes? Beschrijf die zo concreet mogelijk. Gaat het bv.\ om een proof-of-concept, een prototype, een verslag met aanbevelingen, een vergelijkende studie, enz.

\section{\IfLanguageName{dutch}{Opzet van deze bachelorproef}{Structure of this bachelor thesis}}%
\label{sec:opzet-bachelorproef}

% Het is gebruikelijk aan het einde van de inleiding een overzicht te
% geven van de opbouw van de rest van de tekst. Deze sectie bevat al een aanzet
% die je kan aanvullen/aanpassen in functie van je eigen tekst.

De rest van deze bachelorproef is als volgt opgebouwd:

In Hoofdstuk~\ref{ch:stand-van-zaken} wordt een overzicht gegeven van de stand van zaken binnen het onderzoeksdomein, op basis van een literatuurstudie.

In Hoofdstuk~\ref{ch:methodologie} wordt de methodologie toegelicht en worden de gebruikte onderzoekstechnieken besproken om een antwoord te kunnen formuleren op de onderzoeksvragen.

% TODO: Vul hier aan voor je eigen hoofstukken, één of twee zinnen per hoofdstuk

In Hoofdstuk~\ref{ch:conclusie}, tenslotte, wordt de conclusie gegeven en een antwoord geformuleerd op de onderzoeksvragen. Daarbij wordt ook een aanzet gegeven voor toekomstig onderzoek binnen dit domein.
\chapter{\IfLanguageName{dutch}{Stand van zaken}{State of the art}}%
\label{ch:stand-van-zaken}

% Tip: Begin elk hoofdstuk met een paragraaf inleiding die beschrijft hoe
% dit hoofdstuk past binnen het geheel van de bachelorproef. Geef in het
% bijzonder aan wat de link is met het vorige en volgende hoofdstuk.

% Pas na deze inleidende paragraaf komt de eerste sectiehoofding.

Dit hoofdstuk bevat je literatuurstudie. De inhoud gaat verder op de inleiding, maar zal het onderwerp van de bachelorproef *diepgaand* uitspitten. De bedoeling is dat de lezer na lezing van dit hoofdstuk helemaal op de hoogte is van de huidige stand van zaken (state-of-the-art) in het onderzoeksdomein. Iemand die niet vertrouwd is met het onderwerp, weet nu voldoende om de rest van het verhaal te kunnen volgen, zonder dat die er nog andere informatie moet over opzoeken \autocite{Pollefliet2011}.

Je verwijst bij elke bewering die je doet, vakterm die je introduceert, enz.\ naar je bronnen. In \LaTeX{} kan dat met het commando \texttt{$\backslash${textcite\{\}}} of \texttt{$\backslash${autocite\{\}}}. Als argument van het commando geef je de ``sleutel'' van een ``record'' in een bibliografische databank in het Bib\LaTeX{}-formaat (een tekstbestand). Als je expliciet naar de auteur verwijst in de zin (narratieve referentie), gebruik je \texttt{$\backslash${}textcite\{\}}. Soms is de auteursnaam niet expliciet een onderdeel van de zin, dan gebruik je \texttt{$\backslash${}autocite\{\}} (referentie tussen haakjes). Dit gebruik je bv.~bij een citaat, of om in het bijschrift van een overgenomen afbeelding, broncode, tabel, enz. te verwijzen naar de bron. In de volgende paragraaf een voorbeeld van elk.

\textcite{Knuth1998} schreef een van de standaardwerken over sorteer- en zoekalgoritmen. Experten zijn het erover eens dat cloud computing een interessante opportuniteit vormen, zowel voor gebruikers als voor dienstverleners op vlak van informatietechnologie~\autocite{Creeger2009}.

Let er ook op: het \texttt{cite}-commando voor de punt, dus binnen de zin. Je verwijst meteen naar een bron in de eerste zin die erop gebaseerd is, dus niet pas op het einde van een paragraaf.

\lipsum[7-20]

%%=============================================================================
%% Methodologie
%%=============================================================================

\chapter{\IfLanguageName{dutch}{Methodologie}{Methodology}}%
\label{ch:methodologie}

%% TODO: In dit hoofstuk geef je een korte toelichting over hoe je te werk bent
%% gegaan. Verdeel je onderzoek in grote fasen, en licht in elke fase toe wat
%% de doelstelling was, welke deliverables daar uit gekomen zijn, en welke
%% onderzoeksmethoden je daarbij toegepast hebt. Verantwoord waarom je
%% op deze manier te werk gegaan bent.
%% 
%% Voorbeelden van zulke fasen zijn: literatuurstudie, opstellen van een
%% requirements-analyse, opstellen long-list (bij vergelijkende studie),
%% selectie van geschikte tools (bij vergelijkende studie, "short-list"),
%% opzetten testopstelling/PoC, uitvoeren testen en verzamelen
%% van resultaten, analyse van resultaten, ...
%%
%% !!!!! LET OP !!!!!
%%
%% Het is uitdrukkelijk NIET de bedoeling dat je het grootste deel van de corpus
%% van je bachelorproef in dit hoofstuk verwerkt! Dit hoofdstuk is eerder een
%% kort overzicht van je plan van aanpak.
%%
%% Maak voor elke fase (behalve het literatuuronderzoek) een NIEUW HOOFDSTUK aan
%% en geef het een gepaste titel.

\lipsum[21-25]



% Voeg hier je eigen hoofdstukken toe die de ``corpus'' van je bachelorproef
% vormen. De structuur en titels hangen af van je eigen onderzoek. Je kan bv.
% elke fase in je onderzoek in een apart hoofdstuk bespreken.

%\input{...}
%\input{...}
%...

%%=============================================================================
%% Conclusie
%%=============================================================================

\chapter{Conclusie}%
\label{ch:conclusie}

% TODO: Trek een duidelijke conclusie, in de vorm van een antwoord op de
% onderzoeksvra(a)g(en). Wat was jouw bijdrage aan het onderzoeksdomein en
% hoe biedt dit meerwaarde aan het vakgebied/doelgroep? 
% Reflecteer kritisch over het resultaat. In Engelse teksten wordt deze sectie
% ``Discussion'' genoemd. Had je deze uitkomst verwacht? Zijn er zaken die nog
% niet duidelijk zijn?
% Heeft het onderzoek geleid tot nieuwe vragen die uitnodigen tot verder 
%onderzoek?

\lipsum[76-80]



%---------- Bijlagen -----------------------------------------------------------

\appendix

\chapter{Onderzoeksvoorstel}

Het onderwerp van deze bachelorproef is gebaseerd op een onderzoeksvoorstel dat vooraf werd beoordeeld door de promotor. Dat voorstel is opgenomen in deze bijlage.

%% TODO: 
%\section*{Samenvatting}

% Kopieer en plak hier de samenvatting (abstract) van je onderzoeksvoorstel.

% Verwijzing naar het bestand met de inhoud van het onderzoeksvoorstel
%---------- Inleiding ---------------------------------------------------------

\section{Introductie}%
\label{sec:introductie}

Het voorspellen van sportwedstrijden is al lang een zeer populaire bezigheid onder sportfans en gokkers. De uitslagen van deze wedstrijden voorspellen is echter niet altijd even gemakkelijk. Er zijn namelijk veel verschillende factoren die hierin een rol spelen.
Machine Learning (ML), een onderdeel van kunstmatige intelligentie, biedt een potentieel hulpmiddel om de uitslagen van wedstrijden te voorspellen. Door het verzamelen en analyseren van verschillende gegevens, kan een ML model patronen en trends identificeren die dan gebruikt kunnen worden om een mogelijke uitslag te voorspellen.
Het gebruik van ML bij het voorspellen van voetbalwedstrijden is niet nieuw. Toch zijn er nog steeds veel mogelijkheden om dit te verbeteren. In deze bachelorproef wordt een antwoord gegeven op de vraag: “Hoe kan de toepassing van Machine Learning bijdragen tot het voorspellen van voetbalwedstrijden in de Engelse Premier League?”. Om tot dit antwoord te komen worden verschillende deelvragen onderzocht:

\begin{itemize}
  \item	Welke ML techniek is het meest geschikt voor deze toepassing? Supervised, Unsupervised of Reinforced learning.
  \item	Welk algoritme is het meest geschikt voor deze toepassing?
  \item	Welke statistieken hebben een significante invloed op het eindresultaat?
  \item	Hoe kunnen de resultaten van de ML modellen worden geëvalueerd om de nauwkeurigheid en betrouwbaarheid van de voorspellingen te bepalen?
  \item	Wat zijn de beperkingen van ML bij het voorspellen van voetbalwedstrijden en hoe kunnen ze worden opgelost?
  \item	Hoe kan de voorspellende nauwkeurigheid van ML-modellen voor voetbalwedstrijden worden verbeterd?
\end{itemize}

Tijdens dit onderzoek werd gebruikgemaakt van verschillende soorten data: historische wedstrijdresultaten, teamstatistieken, speler statistieken, blessure- en schorsingsrapporten en weersomstandigheden. De historische wedstrijdresultaten en teamstatistieken werden verzameld uit openbare bronnen en databases. De speler statistieken werden eveneens opgehaald uit openbare bronnen en databases. Deze bevatten data zoals: doelpunten, assists, tackles, passes, etc. De blessure- en schorsingsrapporten werden verzameld uit de officiële website van de Engelse Premier League. De weersomstandigheden werden opgehaald uit een meteorologische database en bevat data zoals: temperatuur, neerslag en wind.
Als antwoord op de onderzoeksvraag werd een prototype opgesteld. Dit prototype werd opgesteld op basis van de antwoorden op de gestelde deelvragen.

%---------- Stand van zaken ---------------------------------------------------

\section{Literatuurstudie}%
\label{sec:state-of-the-art}

De afgelopen jaren is de interesse in het toepassen van Machine Learning (ML) in sport toegenomen, zeker bij het voorspellen van voetbalwedstrijden. Deze interesse is deels te wijten aan de groeiende beschikbaarheid van datasets, variërend van speler statistieken tot wedstrijdresultaten. Deze data kan nuttig zijn voor spelers en coaches om hun prestaties te verbeteren, maar ook voor wie wedt op sportevenementen en de bedrijven hierachter.
Om een voetbalwedstrijd te voorspellen moet je rekening houden met veel diverse factoren. \textcite{Stuebinger2020} en \textcite{Rodrigues2022} illustreerden de waarden van gedetailleerde speler statistieken waaronder: lichaamsmetingen, pas nauwkeurigheid, behendigheid, reactie en agressie. Bovendien werd informatie over de prestaties van teams tijdens de wedstrijden - zoals: doelpunten, schoten op kader en aantal hoekschoppen - ook als cruciale data beschouwd. Alhoewel de studie van \textcite{Sathyanarayana2022} over cricket gaat, word hier ook de invloed van weersomstandigheden toegelicht.
Daarnaast werd in de studie van \textcite{Baboota2019} de onvoorspelbaarheid van de sport benadrukt. Ondanks de uitgebreide data-analyse en complexe modellen kunnen er nog steeds verassingen en onverwachte uitkomsten voorkomen, zoals de historische titelzege van Leicester City in 2016.
Tot slot, naast de speler- en teamstatistieken, tonen de studies ook aan dat andere factoren zoals het 'thuisvoordeel' en het verloop van de voorgaande wedstrijd(en) ook een aanzienlijke invloed hebben op het eindresultaat.

Het verfijnen van de data speelt een cruciale rol in het voorspellen van de resultaten met ML. \textcite{Bunker2019} maakte onderscheid tussen 'wedstrijd gerelateerde' en 'externe' factoren. Waarbij deze eerste slaan op de factoren die binnen de wedstrijd vallen zoals: gemaakte meters, passes, enz. Terwijl de 'externe' factoren verwijzen naar de elementen buiten de wedstrijd, zoals recente prestaties en beschikbare spelers.
\textcite{Rodrigues2022} daarentegen creëert nieuwe variabelen om het eindresultaat beter te kunnen voorspellen. Er werden twee soorten variabelen aangemaakt:
\begin{itemize}
  \item het aantal thuisoverwinningen van het thuisspelende team,
  \item het aantal uitoverwinningen van het bezoekende team.
\end{itemize}
Een te grote dataset met veel verschillende features is ook niet al te best om een goede voorspelling te maken. \textcite{Baboota2019} had een totaal van 33 verschillende features waaruit ze de best presterende en meest relevante features haalden aan de hand van feature selection.

Nadat de data verzameld en opgeschoond is, kunnen de modellen getraind worden. Er zijn meerdere ML-modellen die toegepast kunnen worden. Het doel is om te voorspellen of het resultaat een winst (W), gelijkspel (D) of een verlies (L) is voor de thuisploeg, dit wil zeggen dat gebruik gemaakt moet worden van classificatiemodellen.
\textcite{Rodrigues2022} experimenteerde met een breed scala aan modellen, waaronder: Naive Bayes, K-nearest neighbours, random forest, support vector machines, Xgboost en logistic regression. Ook gebuikte hij Artifical Neural Networks, een deep learning model die ook \textcite{Bunker2019}, \textcite{Carloni2021} en \textcite{Azeman2022} gebruikten in hun studies. 
Daarnaast gebruikten \textcite{Stuebinger2020} en \textcite{Andrews2021} ook deze modellen. Stuebinger gebruikte random forest en boosting technieken, deze zijn nuttig voor het omgaan met zowel numerieke als categorische invoer en helpen bij het vermijden van overfitting. Ook gebruikte hij SVM (Support Vector Machines) en Linear Regression, waarbij deze laatste eenvoudig statistisch te interpreteren is.
De hoeveelheid aan modellen gebruikt in de verschillende studies geeft aan dat er een brede keuze is aan ML-modellen voor de voorspelling van voetbalwedstrijden.

Om de modellen nog beter te laten presteren kunnen deze altijd nog geoptimaliseerd worden. Een van de meest voorkomende technieken is het gebruik van een grid search. Dit is een methode waarbij een reeks vooraf gedefinieerde waarden van parameters wordt gebruikt om het optimale model te vinden. Voor elke combinatie van parameters wordt het model getraind en getest om zo de best presterende combinatie te verkrijgen. Meer info hierover  vind je in “Hands-On Machine Learning with Scikit-Learn, Keras, and TensorFlow” door \autocite{Geron2019}.

Nu de modellen geoptimaliseerd en getraind zijn kunnen ze geëvalueerd en vergeleken worden. Hiervoor moeten de juiste prestatie indicatoren gekozen worden die het meest relevant zijn voor het probleem.
\textcite{Stuebinger2020} gebruikte een combinatie van nauwkeurigheid, root mean squared error en de mean absolute deviation om de prestaties van hun modellen te berekenen. Hierbij presteerde het random forest model het beste met de hoogste nauwkeurigheid.
Daarentegen beoordeelde \textcite{Rodrigues2022} de prestaties aan de hand van hun nauwkeurigheid en het percentage correct voorspelde gelijke spelen en overwinningen van zowel de thuisploeg als de uitploeg. In dit onderzoek bleek het SVM model de hoogste nauwkeurigheid te behalen.
\textcite{Azeman2022} gebruikte een uitgebreidere set van prestatie-indicatoren, waaronder: nauwkeurigheid, precisie en recall. Deze indicatoren nemen zowel het aantal correcte voorspellingen als de balans tussen de voorspellingen van positieve en negatieve klassen in overweging.
Er zijn veel verschillende methoden om de prestatie te bereken en er is geen optimale prestatiemaatstaf. Het is belangrijk om de prestatiemaatstaven te kiezen die relevant zijn voor de opdracht.

Op basis van hun uitgebreide studies kwamen de verschillende onderzoekers aan bij differente conclusies.
\textcite{Bunker2019} en \textcite{Andrews2021} benadrukken de noodzaak van meer nauwkeurige ML-modellen in de sportvoorspelling. Dit vooral met het oog op het hoge volume aan sportweddenschappen en de behoefte van sportmanagers aan nuttige informatie voor het ontwikkelen van toekomstige strategieën. Andrews wijst specifiek op het belang van het gebruik van grote datasets voor de training van modellen, gezien de inherente onvoorspelbaarheid van de sport.
Tegelijkertijd tonen de studies van \textcite{Rodrigues2022} en \textcite{Azeman2022} aan dat het testen van verschillende variabelencombinaties succesvol kan zijn. Deze studies leveren modellen op met hoge succespercentages. Rodrigues wees random forest, SVM en Xgboost aan als de beste algoritmen. Azeman vond dat de Multiclass Decision Forest accurater en preciezer was dan het neuraal netwerk en de two-class SVM.
Dergelijke conclusies tonen aan dat machine learning een potentieel krachtig instrument is voor het voorspellen van voetbalwedstrijden. Maar ze tonen ook de uitdagingen die nog bestaan, zoals het juiste model met juiste variabelen kiezen en het rekening houden met een breed scala aan factoren voor het verbeteren van de nauwkeurigheid van de voorspellingen. Ondanks de indrukwekkende vooruitgang in dit gebied is er nog verder onderzoek nodig over dit onderwerp.


% Voor literatuurverwijzingen zijn er twee belangrijke commando's:
% \autocite{KEY} => (Auteur, jaartal) Gebruik dit als de naam van de auteur
%   geen onderdeel is van de zin.
% \textcite{KEY} => Auteur (jaartal)  Gebruik dit als de auteursnaam wel een
%   functie heeft in de zin (bv. ``Uit onderzoek door Doll & Hill (1954) bleek
%   ...'')

%---------- Methodologie ------------------------------------------------------
\section{Methodologie}%
\label{sec:methodologie}

Dit onderzoek start met een  literatuurstudie die verschillende Machine Learning (ML) modellen, technieken en de verschillende prestatiemaatstaven bestudeert.
Na de literatuurstudie haalt een Python script de benodigde data uit diverse bronnen op waardoor één grote dataset met alle benodigde data verkregen wordt. Deze bronnen zijn, onder andere, openbare databases waaruit: historische wedstrijdresultaten, speler statistieken, teamstatistieken en weersomstandigheden opgehaald worden. Daarnaast voorziet de officiële website van de Engelse Premier League blessure- en schorsingsrapporten.
Wanneer alle data in één dataset verzameld zit, kan de data cleaning fase beginnen. Een Python script verwijdert onbelangrijke kolommen en vult ontbrekende gegevens aan. Deze fase bepaalt welke factoren de grootste invloed hebben op de uiteindelijke uitslag en bekijkt of combinaties van verschillende factoren een sterkere factor kunnen creëren. Ten slotte, wordt de dataset opgesplitst in een trainings- en testset om de prestatie van de ML-modellen te berekenen.
Nadat de data cleaning fase voltooid is, trainen de ML-modellen op de trainingsdataset. Omdat er voorspeld wordt of de wedstrijd eindigt op een winst (W), gelijkspel (D) of verlies (L), wordt gebruikgemaakt van classificatiemodellen. Decision Tree, Random Forrest en Support vector machines zijn een paar voorbeelden van de verschillende te testen classificatie modellen. Deep learning modellen zoals neurale netwerken komen ook aan bod. Hierna volgt een optimalisatiefase die de beste parameters aan elk model geeft. 
Eens alle modellen getraind en geoptimaliseerd zijn kunnen deze geëvalueerd worden. Dit aan de hand van verschillende prestatiemaatstaven, zoals: nauwkeurigheid (accuracy), precision, recall, Z1-score, Confusion Matrix en andere. De ML-modellen krijgen een grondige evaluatie op basis van het bekomen, uitgebreide overzicht aan prestatiemaatstaven. 
Als laatste stap worden conclusies getrokken over de bekomen resultaten. Er wordt bekeken welke factoren het meest doorwegen op het eindresultaat en welke ML-modellen het beste presteren.

%---------- Verwachte resultaten ----------------------------------------------
\section{Verwacht resultaat, conclusie}%
\label{sec:verwachte_resultaten}

Uit het onderzoek wordt verwacht dat Machine Learning (ML) een mogelijk hulpmiddel kan zijn bij het voorspellen van voetbalwedstrijden. Omdat voetbal een zeer onvoorspelbare sport is zullen de resultaten nooit 100\% accuraat zijn. De resultaten kunnen wel een goede indicatie zijn naar een mogelijke uitslag van de wedstrijd.
De factoren die vermoedelijk het meeste invloed op het eindresultaat zullen hebben zijn de 'historische wedstrijduitslagen' en het 'thuisvoordeel'. Dit is logisch aangezien resultaten van vorige wedstrijden een goede indicator kunnen zijn naar een volgende wedstrijd. Daarenboven speelt de thuisploeg vaker in eigen stadion en heeft zij meer aanwezige supporters.
De factor weersomstandigheden daarentegen zal geen aanzienlijke rol spelen. Deze factor kan het spelbeeld van een wedstrijd zeker beïnvloeden maar zal geen significante invloed hebben op de uitslag.
Het deep learning model neurale netwerken wordt voorspelt het beste te presteren. Dit model is zeer complex en zeer goed in patronen herkennen. Wat bij deze dataset extra belangrijk is aangezien er veel verschillende factoren zijn die een rol spelen bij de voorspellingen.
De nauwkeurigheid van de voorspellingen wordt verwacht een percentage tussen de 65\% en 70\% te hebben. Dit is zeer goed aangezien voetbal een onvoorspelbare sport is. 


%%---------- Andere bijlagen --------------------------------------------------
% TODO: Voeg hier eventuele andere bijlagen toe. Bv. als je deze BP voor de
% tweede keer indient, een overzicht van de verbeteringen t.o.v. het origineel.
%\input{...}

%%---------- Backmatter, referentielijst ---------------------------------------

\backmatter{}

\setlength\bibitemsep{2pt} %% Add Some space between the bibliograpy entries
\printbibliography[heading=bibintoc]

\end{document}
